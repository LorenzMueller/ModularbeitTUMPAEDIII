%%%%%%%%%%%%%%%%%%%%%%%%%%%%%%%%%%%%%%%%%%%%%%%%%%%%%%%%%%%%%%%%%%%%%%%%%%%%%%%%
% TUM-Vorlage: Wissenschaftliche Arbeit
%%%%%%%%%%%%%%%%%%%%%%%%%%%%%%%%%%%%%%%%%%%%%%%%%%%%%%%%%%%%%%%%%%%%%%%%%%%%%%%%
%
% Rechteinhaber:
%     Technische Universität München
%     https://www.tum.de
% 
% Gestaltung:
%     ediundsepp Gestaltungsgesellschaft, München
%     http://www.ediundsepp.de
% 
% Technische Umsetzung:
%     eWorks GmbH, Frankfurt am Main
%     http://www.eworks.de
%
%%%%%%%%%%%%%%%%%%%%%%%%%%%%%%%%%%%%%%%%%%%%%%%%%%%%%%%%%%%%%%%%%%%%%%%%%%%%%%%%


%%%%%%%%%%%%%%%%%%%%%%%%%%%%%%%%%%%%%%%%%%%%%%%%%%%%%%%%%%%%%%%%%%%%%%%%%%%%%%%%
\documentclass[%
    fontsize=11pt, % Schriftgröße
    twoside=off % kein einseitiges Layout
]{scrbook} % Dokumentenklasse: KOMA-Script Book
\usepackage{scrlayer-scrpage} % Anpassbare Kopf- und Fußzeilen

\usepackage[utf8]{inputenc} % Textkodierung: UTF-8
\usepackage[T1]{fontenc} % Zeichensatzkodierung

\usepackage[german]{babel} % Deutsche Lokalisierung
\usepackage{graphicx} % Grafiken
% Schriftart Helvetica:
\usepackage[scaled]{helvet}
\renewcommand{\familydefault}{\sfdefault}


\usepackage[hidelinks]{hyperref} % Hyperlinks
\usepackage[onehalfspacing]{setspace} % 1,5facher Zeilenabstand
\usepackage{calc} % Berechnungen


\usepackage{tabularx} % Flexiblere Tabellen
\usepackage{caption} % Anpassen von Beschriftungen

% Nummerierung von Abbildungen & Tabellen durchgängig, statt nach Kapiteln:
\usepackage{chngcntr}
\counterwithout{figure}{chapter}
\counterwithout{table}{chapter}

% Abkürzungen, Glossare:
\usepackage[%
    acronym,% Separates Akronym-Verzeichnis
    nopostdot,% Kein Punkt am Ende einer Beschreibung im Glossar
]{glossaries}

% Spezielle Befehlsdefinitionen:
\newcommand{\Thema}{}
% Debugging:
%\usepackage{showframe} % Layout-Boxen anzeigen
%\usepackage{layout} % Layout-Informationen
%\usepackage{printlen} % Längenwerte ausgeben
 % !!! NICHT ENTFERNEN !!!
%%%%%%%%%%%%%%%%%%%%%%%%%%%%%%%%%%%%%%%%%%%%%%%%%%%%%%%%%%%%%%%%%%%%%%%%%%%%%%%%

\renewcommand{\Thema}{%
    Projektarbeit DFA}

%%%%%%%%%%%%%%%%%%%%%%%%%%%%%%%%%%%%%%%%%%%%%%%%%%%%%%%%%%%%%%%%%%%%%%%%%%%%%%%%
\input{./Ressourcen/Anfang.tex} % !!! NICHT ENTFERNEN !!!
%%%%%%%%%%%%%%%%%%%%%%%%%%%%%%%%%%%%%%%%%%%%%%%%%%%%%%%%%%%%%%%%%%%%%%%%%%%%%%%%

\begin{document}



\title{Projektarbeit DFA}
\author{Lorenz Müller}
\date{16.07.2018}


\section*{Abstract}

\textit{Menschen lernen offensichtlich in unterschiedlicher Art und Weise. Eine solche Unterteilung kann zum einen in Persönlichkeitstypen, wie bei Schrader, als auch in kompetenzbezogene Unterschiede, wie bei Zech, getroffen werden. Zum anderen können Aufgaben anhand ihres Grades an Nützlichkeit, in Form von beigelegten Bildern, gestaltet sein. Diese Bilder können entweder nur dekorativen Nutzen haben oder auch essenziell für das Lösen der Aufgabe sein und somit zusätzlich eine Anschauung vermitteln. In vorliegender Studie wurden 39 Studenten der TU München in einer Eyetrackingstudie zu den Fragestellungen untersucht: Lassen sich die Studenten in Lerntypen anhand ihrer Augenbewegungen einteilen? Wie wirkt sich der Lerntyp auf die Leistungsfähigkeit der Versuchsteilnehmer aus? Mit welcher Erfolgsquote werden dekorative oder essenzielle Bilder bearbeitet? Und schließlich, gibt es Lerntypen, die mit unterschiedlich beigelegten Bildern, besser oder schlechter umgehen können?}

\textit{Um diese Fragen beantworten zu können, wurde den Studenten zu Beginn ein heuristisches Lösungsbeispiel für einen mathematischen Zusammenhang gezeigt. Die Blickbewegungen der Studenten auf diesem Lösungsbeispiel legt die Lerntypunterteilung der Probandinnen und Probanden fest. Im weiteren Teil wurden mathematische Aufgaben mit dekorativen und essenziellen Bildern gestellt, welche die Probandinnen und Probanden bearbeiten mussten. Eine Unterteilung der Lerntypen war in der Mehrheit der Versuchsteilnehmer möglich, jedoch die Unterschiede ihrer Leistungsfähigkeit nicht signifikant. Bei der Verwendung unterschiedlicher Bildtypen wurden Aufgaben mit essenziellen Bildern etwas besser bearbeitet als Aufgaben mit nur dekorativen Bildern. Bei der Auswertung, wie die unterschiedlichen Lerntypen mit den beigefügten Bildern umgegangen sind, war festzustellen, dass Lernende, die im ersten Teil der Studie sehr oft die Abbildungen betrachtet haben, im zweiten Teil mit essenziellen Bildern nicht gewinnbringend umgehen konnten.}


\tableofcontents % Inhaltsverzeichnis

\chapter{Einleitung}
Im ersten Schritt des Projektes stand im Vordergrund, welches Projekt man überhaupt realisieren wollte. Hierbei kristalisierte sich schnell herraus, dass ein Projekt sinvoll erschien, welches sich mit gerade gelernten Objekten befasst. Da Elena, Philip und ich alle im Semester davor die Vorlesung Theoretische Informatik gehört hatten, reizte es und zu diesem Thema einen DFA-Editor zu programmieren, welcher verschiedene Funktionen für den User überprüft und veranschaulicht. 
\chapter{Lasten- und Pflichtenhäft}
\input{Tex/UMLGraphiken.tex}



\printbibliography
\clearpage

\listoffigures % Abbildungsverzeichnis

\printacronyms[title={Abkürzungsverzeichnis}] % Abkürzungsverzeichnis

\listoftables % Tabellenverzeichnis

\onehalfspacing
\end{document}