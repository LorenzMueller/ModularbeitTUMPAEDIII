%%%%%%%%%%%%%%%%%%%%%%%%%%%%%%%%%%%%%%%%%%%%%%%%%%%%%%%%%%%%%%%%%%%%%%%%%%%%%%%%
% TUM-Vorlage: Wissenschaftliche Arbeit
%%%%%%%%%%%%%%%%%%%%%%%%%%%%%%%%%%%%%%%%%%%%%%%%%%%%%%%%%%%%%%%%%%%%%%%%%%%%%%%%
%
% Rechteinhaber:
%     Technische Universität München
%     https://www.tum.de
% 
% Gestaltung:
%     ediundsepp Gestaltungsgesellschaft, München
%     http://www.ediundsepp.de
% 
% Technische Umsetzung:
%     eWorks GmbH, Frankfurt am Main
%     http://www.eworks.de
%
%%%%%%%%%%%%%%%%%%%%%%%%%%%%%%%%%%%%%%%%%%%%%%%%%%%%%%%%%%%%%%%%%%%%%%%%%%%%%%%%


%%%%%%%%%%%%%%%%%%%%%%%%%%%%%%%%%%%%%%%%%%%%%%%%%%%%%%%%%%%%%%%%%%%%%%%%%%%%%%%%
\documentclass[%
    fontsize=11pt, % Schriftgröße
    twoside=off % kein einseitiges Layout
]{scrbook} % Dokumentenklasse: KOMA-Script Book
\usepackage{scrlayer-scrpage} % Anpassbare Kopf- und Fußzeilen

\usepackage[utf8]{inputenc} % Textkodierung: UTF-8
\usepackage[T1]{fontenc} % Zeichensatzkodierung

\usepackage[german]{babel} % Deutsche Lokalisierung
\usepackage{graphicx} % Grafiken
% Schriftart Helvetica:
\usepackage[scaled]{helvet}
\renewcommand{\familydefault}{\sfdefault}


\usepackage[hidelinks]{hyperref} % Hyperlinks
\usepackage[onehalfspacing]{setspace} % 1,5facher Zeilenabstand
\usepackage{calc} % Berechnungen


\usepackage{tabularx} % Flexiblere Tabellen
\usepackage{caption} % Anpassen von Beschriftungen

% Nummerierung von Abbildungen & Tabellen durchgängig, statt nach Kapiteln:
\usepackage{chngcntr}
\counterwithout{figure}{chapter}
\counterwithout{table}{chapter}

% Abkürzungen, Glossare:
\usepackage[%
    acronym,% Separates Akronym-Verzeichnis
    nopostdot,% Kein Punkt am Ende einer Beschreibung im Glossar
]{glossaries}

% Spezielle Befehlsdefinitionen:
\newcommand{\Thema}{}
% Debugging:
%\usepackage{showframe} % Layout-Boxen anzeigen
%\usepackage{layout} % Layout-Informationen
%\usepackage{printlen} % Längenwerte ausgeben
 % !!! NICHT ENTFERNEN !!!
%%%%%%%%%%%%%%%%%%%%%%%%%%%%%%%%%%%%%%%%%%%%%%%%%%%%%%%%%%%%%%%%%%%%%%%%%%%%%%%%

\renewcommand{\Thema}{%
   Addieren natürlicher Zahlen}

%%%%%%%%%%%%%%%%%%%%%%%%%%%%%%%%%%%%%%%%%%%%%%%%%%%%%%%%%%%%%%%%%%%%%%%%%%%%%%%%
%%%%%%%%%%%%%%%%%%%%%%%%%%%%%%%%%%%%%%%%%%%%%%%%%%%%%%%%%%%%%%%%%%%%%%%%%%%%%%%%
% EINSTELLUNGEN
%%%%%%%%%%%%%%%%%%%%%%%%%%%%%%%%%%%%%%%%%%%%%%%%%%%%%%%%%%%%%%%%%%%%%%%%%%%%%%%%

\KOMAoptions{parskip=full}


% Seitenränder:

\newcommand{\SeitenrandOben}{25.8mm}
\newcommand{\SeitenrandRechts}{21mm}
\newcommand{\SeitenrandLinks}{40mm}
\newcommand{\SeitenrandUnten}{24.8mm}
\newcommand{\FusszeileHoehe}{11.7mm}

\usepackage[a4paper,
    head=0pt,
    top=\SeitenrandOben,
    bottom=\SeitenrandUnten,
    inner=\SeitenrandLinks,
    outer=\SeitenrandRechts
]{geometry}


% Fußzeilen:

\setlength{\footheight}{\FusszeileHoehe}
\clearscrheadfoot
\ifoot*{\Thema\vfill}
\ofoot*{\pagemark\vfill}
\setkomafont{pageheadfoot}{\fontsize{9pt}{13pt}\normalfont}
\setkomafont{pagefoot}{\bfseries}
\setkomafont{pagenumber}{\normalfont}
\pagestyle{scrheadings}


% Fußnoten:

\KOMAoptions{%
    footnotes=multiple % mehrere Fußnoten werden durch Zeichen getrennt
}
%\setfootnoterule[.6pt]{5.08cm}
\renewcommand{\footnoterule}{\hrule width 5.08cm height .6pt \vspace*{3.9mm}}
%\setlength{\footnotesep}{5mm}
\deffootnote{2mm}{2mm}{%
    \makebox[2mm][l]{\textsuperscript{\thefootnotemark}}%
}
\setkomafont{footnoterule}{\fontsize{9pt}{20pt}\selectfont}


% Überschriften:

\KOMAoptions{%
    open=any, % keine Festlegung auf linke oder rechte Seite
    numbers=noendperiod, % kein automatischer Punkt nach Gliederungsnummer
    headings=small
}

\makeatletter
\g@addto@macro{\@afterheading}{\vspace{-\parskip}} % \parskip nach Gliederungsbefehlen entfernen
\renewcommand*{\chapterheadstartvskip}{\vspace{\@tempskipa}\vspace{-3pt}} % Korrektur für Abstand über Kapitelüberschriften
\makeatother

\setkomafont{disposition}{\normalfont\sffamily}

\setkomafont{chapter}{\normalfont\fontsize{19pt}{22pt}\selectfont}
\RedeclareSectionCommand[%
  beforeskip=0pt,
  afterskip=29pt
]{chapter}
\renewcommand*{\chapterformat}{\thechapter.\enskip} % Immer Punkt nach Kapitelnummer

\setkomafont{section}{\fontsize{15pt}{17pt}\selectfont}
\RedeclareSectionCommand[%
  beforeskip=0pt,
  afterskip=24.1pt
]{section}
\renewcommand*{\sectionformat}{\makebox[13mm][l]{\thesection.\enskip}} % Feste Breite für Abschnittsnummer und immer Punkt danach

\setkomafont{subsection}{\bfseries\fontsize{12pt}{13pt}\selectfont}
\RedeclareSectionCommand[%
  beforeskip=0pt,
  afterskip=1pt
]{subsection}
\renewcommand*{\subsectionformat}{\makebox[13mm][l]{\thesubsection.\enskip}} % Feste Breite für Unterabschnittsnummer und immer Punkt danach

\usepackage{csquotes}

\usepackage[backend=biber, style=apa]{biblatex}
\addbibresource{Bib.bib}

\usepackage{float}
% Listen:
\usepackage{enumitem}

\setlist{%
    labelsep=0mm,
    itemindent=0pt,
    labelindent=0pt,
    align=left,
    parsep=1.5ex
}
\setlist[itemize]{%
    leftmargin=5mm,
    labelwidth=4.9mm
}
\setlist[itemize,1]{%
    before={\vspace{0.25ex}},
    label={\raisebox{.35ex}{\textbullet}},
    after={\vspace{-\parsep}\vspace{-.25ex}}
}
\setlist[itemize,2]{%
    label={\raisebox{.35ex}{\rule{.58ex}{.58ex}}}
}
\setlist[enumerate]{%
    leftmargin=10mm,
    labelwidth=9.9mm
}
\setlist[enumerate,2]{%
    label={\alph*.}
}

\setlist[description]{%
%    labelindent=!,
    leftmargin=1em,
    labelwidth=!,
    parsep=0mm,
    partopsep=0mm,
    labelsep=1em,
}


% Verzeichnisse:

\KOMAoptions{%
    toc=flat, % keine Einrückungen im Inhaltsverzeichnis
    toc=chapterentrydotfill, % Punkte bis zur Seitennummer bei Kapiteln
    listof=entryprefix, % Präfix für Einträge in Abbildungs- und Tabellenverzeichnis
    listof=nochaptergap, % Kein Abstand für Kapiteleinträge in extra Verzeichnissen
}

\makeatletter
\renewcommand{\@dotsep}{.3} % Abstand der Füllpunkte

% "chapteratlist" für Inhaltsverzeichnis auswerten:
\renewcommand*{\addchaptertocentry}[2]{%
  \iftocfeature{toc}{chapteratlist}{}{%
    \addtocontents{toc}{\protect\vspace{-10pt}}% extra Abstand vor Kapitelüberschriften in Inhaltsverzeichnis entfernen
  }%
  % Originaldefinition aus scrbook.cls:
  \addtocentrydefault{chapter}{#1}{#2}%
  \if@chaptertolists
    \doforeachtocfile{%
      \iftocfeature{\@currext}{chapteratlist}{%
        \addxcontentsline{\@currext}{chapteratlist}[{#1}]{#2}%
      }{}%
    }%
    \@ifundefined{float@addtolists}{}{\scr@float@addtolists@warning}%
  \fi%
}
\makeatother

\AfterTOCHead[toc]{\protect\vspace{.8ex}} % Abstand zwischen Überschrift und Inhaltsverzeichnis
\setuptoc{toc}{noparskipfake} % Angleichung der Abstände nach Inhaltsverzeichnisüberschrift an andere Überschriften
\unsettoc{toc}{chapteratlist} % kein Abstand vor Kapiteleinträgen im Inhaltsverzeichnis, funktioniert nur durch obige Redefinition von \addchaptertocentry

% -- Abbildungs- und Tabellenverzeichnis:

\AfterTOCHead[lof]{\protect\vspace{-.1ex}\doublespacing} % Abstand zwischen Überschrift und Abbildungsverzeichnis, doppelter Zeilenabstand
\setuptoc{lof}{noparskipfake} % Angleichung der Abstände nach Abbildungsverzeichnisüberschrift an andere Überschriften

\AfterTOCHead[lot]{\protect\vspace{-.1ex}\doublespacing} % Abstand zwischen Überschrift und Tabellenverzeichnis, doppelter Zeilenabstand
\setuptoc{lot}{noparskipfake} % Angleichung der Abstände nach Tabellenverzeichnisüberschrift an andere Überschriften

% Beschriftungen:
\DeclareCaptionFormat{WissenschaftlicheArbeiten}{\fontsize{8pt}{10pt}\selectfont#1 #2#3\par}
\DeclareCaptionLabelFormat{WissenschaftlicheArbeiten}{\bfseries\selectfont#1 #2}

% -- Tabellen:
\captionsetup[table]{%
    format=WissenschaftlicheArbeiten,
    labelformat=WissenschaftlicheArbeiten,
    labelsep=none,
    singlelinecheck=off,
    justification=raggedright,
    skip=3pt,
    tablewithin=none
}

% -- Abbildungen:
\captionsetup[figure]{%
    format=WissenschaftlicheArbeiten,
    labelformat=WissenschaftlicheArbeiten,
    labelsep=none,
    singlelinecheck=off,
    justification=raggedright,
    skip=6.6mm,
    figurewithin=none
}


% Tabellen:
\renewcommand{\arraystretch}{1.8} % Skalierung der Tabellen
\newcolumntype{M}{X<{\vspace{4pt}}} % Spaltentyp mit Abstand rechts


% Glossare & Abkürzungsverzeichnis:

\makeglossaries
\newacronym{abk}{Abk.}{Abkürzungen}
\newacronym{SuS}{SuS}{Schülerinner und Schüler}
\newacronym{Gr1}{Gr1.}{Typ Unsicher}
\newacronym{Gr2}{Gr2.}{Typ Problemlöser}
\newacronym{Gr3}{Gr3.}{Typ Textuell}
\newacronym{Rad}{ARad.}{Aufgabe Radfahrerin}
\newacronym{Ren}{ARen.}{Aufgabe Rennfahrer}
\newacronym{Zuf}{AZuf.}{Aufgabe Zufall}
\newacronym{Ber}{ABerg.}{Aufgabe Bergsteigen}
\newacronym{Aut}{AAut.}{Aufgabe Autofahren}
\newacronym{PuP}{PuP}{Probandinnnen und Probanden}
\newacronym{dPodP}{dPodP}{die Probandin oder der Proband}


\setacronymstyle{short-long}

\makeatletter
\newlength{\@glsdotsep}
\setlength{\@glsdotsep}{\@dotsep em}
\newcommand*{\glsdotfill}{\leavevmode \cleaders \hb@xt@ \@glsdotsep{\hss .\hss }\hfill \kern \z@}
\makeatother

\newglossarystyle{WissenschaftlicheArbeiten}{%
  \setglossarystyle{index}%

  \renewcommand*{\glossaryheader}{\vspace{.75em}}%
  \renewcommand*{\glstreenamefmt}[1]{##1}%
  \renewcommand*{\glossentry}[2]{%
     \item\glsentryitem{##1}\glstreenamefmt{\glstarget{##1}{\glossentryname{##1}}}%
     \ifglshassymbol{##1}{\space(\glossentrysymbol{##1})}{}%
     \space-\space\glossentrydesc{##1}\glsdotfill\glspostdescription\space ##2%
  }%
  \renewcommand*{\glsgroupheading}[1]{%
    \item\glstreenamefmt{\textbf{\fontsize{14}{17}\selectfont\enskip\glsgetgrouptitle{##1}}}\vspace{.3em}}%
}

\setglossarystyle{WissenschaftlicheArbeiten}




 % !!! NICHT ENTFERNEN !!!
%%%%%%%%%%%%%%%%%%%%%%%%%%%%%%%%%%%%%%%%%%%%%%%%%%%%%%%%%%%%%%%%%%%%%%%%%%%%%%%%
\usepackage{amssymb} 
\begin{document}



\title{Unterrichtsstunde: Addieren natürlicher Zahlen}
\author{Lorenz Müller}
\date{16.07.2018}




\tableofcontents % Inhaltsverzeichnis

\chapter{Formalia}
Im Rahmen des TUM-Paedagogicum III – Umgang mit Heterogenität im Fachkontext –
welches das Seminar „Innere Differenzierung/Adaptiver Unterricht/Selbstreguliertes Lernen“
unter der Leitung von Frau Dr. Jutta Möhringer, sowie das Seminar zum studienbegleitenden
fachdidaktischen Praktikum im Fach Mathematik unter der Leitung von Herrn Frank Reinhold
und das Semesterbegleitenden Schulpraktikum umfasst, wurde diese Modularbeit erstellt.
\\
\\
Diese Arbeit beschäftigt sich mit der Planung, der Durchführung und der Auswertung einer Unterrichtsstunde im Fach Mathematik. Diese Unterrichtsstunde wurde am Rupprecht Gymnasium in München am 09.11.2018 gehalten.
\\
\\
Das Rupprecht Gymnasium wird von etwa 1100 SuS besucht, wobei mehr als 80 Lehrerinnen und Lehrer beteiligt sind. Das Gymnasium liegt im Stadtteil Neuhausen und bietet sowohl einen sprachlichen (Französisch oder Italienisch als dritte Fremdsprache) als auch einen naturwissenschaftlich-technologischen Zweig mit den Schwerpunkfächern Informatik und Chemie an. Ebenso werden an außerschulischen Projekten, wie Physik im Advent mit einzelnen Klassenstufen teilgenommen.
\\
\\
Die Unterrichtsstunde wurde in der fünften Stunde von 11:20 bis 12:05 im Raum 101 gehalten. Zusätzlich zur Klasse 5b waren Frau Judith Polz und ein Kommiliton anwesend. Die Klasse hatte vor der Unterrichtseinheit zwei Stunden Deutsch eine Stunde Englisch und eine Stunde Kunst.
\chapter{Stundenthema}
Das Thema der Stunde war \glqq Addieren von natürlichen Zahlen \grqq. Diese Unterrichtsstunde lässt sich in die Unterrichtssequenz Natürliche und ganze Zahlen - Addition und Subraktion einbetten, welche am Anfang der 5 Jahrgangsstufe steht. In der vorheringen Stunde wurden negative Zahlen behandelt. Diese Unterrichtsstunde zieht sich somit wieder auf die näturlichen Zahlen zurück und wiederholt bereits in der Grundschule gelernte Kompetenzen. Weitere Infromationen zu diesem Kontext sind im Abschnitt \glqq Einordnung der Unterrichtseinheit \grqq in die Lernsequenz und Bezug zum Lehrplan zu finden
\chapter{Beobachtungsschwerpunkt}
\chapter{Analyse der individuellen und soziokulturellen Voraussetzungen}

\section{Zusammensetzung der Klasse}
Die Klasse 5a besteht aus 16 Schülern und und 18 Schülerinnen, also insgesamt aus 32 Kindern. Alle Kinder der Klasse beherschen gutes Deutsch, somit gibt es kaum sprachliche Barieren. Die noch sehr jungen SuS benötigen eine regelrechte Einführung in das Schulleben am Gymnasium. Dies bedeutet, dass ein Großteil der Stunde verlohren gehen kann, weil man sich mit Fragen wie: \glqq Ich habe nicht die gleiche Farbe wie Sie, was soll ich jetzt tuen? \grqq. Hierbei wurde zum einen eine gesamte Schulstunde darauf verwendet, wie mit Formatierungen an der Tafel (Farben, Unterstreichen, Häftorientierung) umgegangen werden soll, als auch einzelne SuS persönlich gebeten nur noch relevante Unterrichtsbeiträge in das Unterrichtsgespräch einzubringen.\\
\section{Lernklima und Interaktionsverhalten}
Die SuS sind sehr motiviert am Unterrichtsgeschehen teilhaben zu können. Die Lehrerin Frau Polz spielt mit ihnen jede Stunde ein kleines Tafelspiel, wobei zwei Striche und ein Kreis an die Tafel gemalt werden. Wenn die Klasse unruhiger wird kommt erst der eine Strich, dann der andere Strich weg. Sollte es danach nochmals unruhig werden kommt der Kreis auch weg. Sollte es die Klasse schaffen den Kreis bis zum Ende der Stunde zu behalten, kriegen sie weniger oder keine Hausaufgaben. Dieses Spiel wird von der Klasse sehr ernst genommen und die SuS ermahnen sich regelmäßig untereinander ruhig zu sein. Dieses Spiel garantiert in dieser Klasse, dass das Lernklima meistens vorbildlich ist. \\
In den Arbeitsphasen ist des den SuS nicht gestattet sich mit ihrem Nachbarn zu unterhalten, um außerschulischen Gesprächen vorzubeugen. Es ist den SuS aber gestattet aufzustehen und jemand anderen aus der Klasse um Hilfe zu bitten, wenn man bei einer Aufgabe nicht weiter kommt. Dieses Angebot wird von wenigen Lernenden genutzt und kaum missbraucht, meist gehen die Fragen an die Lehrkraft. 
\section {Lernumgebung}
Im Klassenzimmer befindent zusätzlich zur normalen Tafel eine Dokumentenkammera mit Beamer, welche bei den hohen Wänden sehr gut auch über der Tafel genutzt werden kann. Somit können Bilder oder Ähnliches über die Dokumentenkammera gleichzeitig mit dem Tafelbild den SuS vorgeführt werden. Die SuS haben an der hinteren Wand vom Klassenzimmer ein Profil von sich selbst aufgehangen. Diese Profil zeigt was das betreffende Kind gerne in der Freizeig unternimmt, das Alter und die Lieblingsspeise. Dies steigert die Verbundenheit zur Lernumgebung. (Studie?)
\chapter{Sachanalyse}
\chapter{Einordnung der Unterrichtseinheit in die Lernsequenz und Bezug zum Lehrplan}
Diese Unterrichtsstunde lässt sich in die Unterrichtssequenz M5.1 Natürliche und ganze Zahlen - Addition und Subraktion einbetten, welche 30 Unterrichtsstunden umfasst. 
\chapter{Lernziele}
\chapter{Didaktische Analyse}
\chapter{Methodische Analyse}
\chapter{Artikulationsschema}


\begin{tabularx}{\textwidth}{|X|X|X|X|X|}
\hline
\textbf{Zeit/Sozialform} & \textbf{Geplantes Lehrerverhalten}& \textbf{Geplantes Schülerverhalten} & \textbf{Medien, Materialien} & \textbf{Anmerkungen}\\
\hline
\hline
\multicolumn{5}{|l|}{Wiederholung der letzten Stunde}\\
\hline
5 Minuten, Plenum        &
Der Lehrer kontroliert die Hausaufabe, welche von einem SoS vorgestellt wird. &
Die SuS kontrollieren, ob sie ihre Hausaufgabe richtig bearbeitet haben. Ebenso können bei unterschiedlichen Antworten die Lehrkraft zu Rate gezogen werden, wieso hieber Unterschiede vorhanden sind &
Dokumenten- kammera, Hausaufgabe der letzten Stunde & 7\\
\hline
\multicolumn{5}{|l|}{Motivationsphase (Einstieg und Hinführung)}\\
\hline
7 Minuten, Plenum &
Der Lehrer legt die Abbildung des Baumes und der Schneke unter die Dokumenten- kammera. Hierbei wird den SuS der Sachzusammenhang erklärt und ein SoS gebeten die Bewegung zu veranschaulichen. &
Ein SoS präsentiert ihre Lösung den anderen SuS, die restlichen SuS verfolgen den Verlauf der Schneke und helfen bei der Lösung mit. &
Dokumenten- kammera, Baum- und Schnekenabbilung  & 7\\
\hline
\multicolumn{5}{|l|}{Wiedeholungsphase (durch einfache Beispiele)}\\
\hline
5 Minuten, Plenum &
Die Lehrkraft berechnent mit den SuS eine Addition und eine Subtraktion einer dreistelligen Zahl. &
Die SuS erinnern sich an die oder das gelernte Verfahren, aus der Grundschule, welches sie anwenden können um eine Additon oder eine Subtraktion schriftlich untereinander durchzuführen &
Tafel & 7\\
\multicolumn{5}{|l|}{Verallgemeinerungs- und Ergebnissicherungsphase }\\
\hline
10 Minuten, Plenum &
Der Lehrer schreibt den Häfteintrag an die Tafel und veranschaulicht den Zusammenhang zwischen dem Einführungsbeispiel der Schneke mit dem Häfteintrag&
Die SuS schreiben den Häfteintrag ab. Ebenso sollte durch den Häfteintrag mit der Verbindung der Wiederholungs und der Motivationsphase ein besseres Verständniss für den Zusammenhang zwischen Zahlenstrahl und einfacher Rechung entstanden sein. Zusätzlich werden Fachbegriffe der Addition und der Subtraktion gelernt. &
Tafel, Häfteintrag & 7\\
\multicolumn{5}{|l|}{Anwendungsphase}\\
\hline
13 Minuten, Partnerarbeit &
Der Lehrer erklärt die Aufgabenstellung und unterstützt die SuS bei Fragen.&
Die SuS bearbeiten mit ihrem Nachbarn zusammen die gegebenen Aufgaben. Hierbei werden sowohl die Fachbegriffe  als auch die Rechentechniken vertieft.&
Tafel, Buch & 7\\
\multicolumn{5}{|l|}{Sicherungsphase}\\
\hline
5 Minuten, Plenum &
Der Lehrer lässt einzelne SuS zusammenfassen, was Thema der Stunde war und assistiert hierbei.&
Die SuS festigen ihr, in dieser Stunde erworbenens Wissen, durch die Wiederholung im Plenum &
 & 7\\
\end{tabularx}


\chapter{Reflexion}

\printbibliography
\clearpage

\listoffigures % Abbildungsverzeichnis

\printacronyms[title={Abkürzungsverzeichnis}] % Abkürzungsverzeichnis

\listoftables % Tabellenverzeichnis

\onehalfspacing
\end{document}