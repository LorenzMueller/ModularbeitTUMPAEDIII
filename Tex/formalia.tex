\chapter{Formalia}
Im Rahmen des TUM-Paedagogicum III – Umgang mit Heterogenität im Fachkontext –
welches das Seminar „Innere Differenzierung/Adaptiver Unterricht/Selbstreguliertes Lernen“
unter der Leitung von Frau Dr. Jutta Möhringer, sowie das Seminar zum studienbegleitenden
fachdidaktischen Praktikum im Fach Mathematik unter der Leitung von Herrn Frank Reinhold
und das Semesterbegleitenden Schulpraktikum umfasst, wurde diese Modularbeit erstellt.
\\
\\
Diese Arbeit beschäftigt sich mit der Planung, der Durchführung und der Auswertung einer Unterrichtsstunde im Fach Mathematik. Diese Unterrichtsstunde wurde am Rupprecht Gymnasium in München am 09.11.2018 gehalten.
\\
\\
Das Rupprecht Gymnasium wird von etwa 1100 SuS besucht, wobei mehr als 80 Lehrerinnen und Lehrer beteiligt sind. Das Gymnasium liegt im Stadtteil Neuhausen und bietet sowohl einen sprachlichen (Französisch oder Italienisch als dritte Fremdsprache) als auch einen naturwissenschaftlich-technologischen Zweig mit den Schwerpunkfächern Informatik und Chemie an. Ebenso werden an außerschulischen Projekten, wie Physik im Advent mit einzelnen Klassenstufen teilgenommen.
\\
\\
Die Unterrichtsstunde wurde in der fünften Stunde von 11:20 bis 12:05 im Raum 101 gehalten. Zusätzlich zur Klasse 5b waren Frau Judith Polz und ein Kommiliton anwesend. Die Klasse hatte vor der Unterrichtseinheit zwei Stunden Deutsch eine Stunde Englisch und eine Stunde Kunst.