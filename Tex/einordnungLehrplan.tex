\chapter{Einordnung der Unterrichtseinheit in die Lernsequenz und Bezug zum Lehrplan}
Diese Unterrichtsstunde lässt sich in die Unterrichtssequenz \textbf{M5.1 Natürliche und ganze Zahlen - Addition und Subraktion} einbetten, welche 30 Unterrichtsstunden umfasst. In dem verwendetet Schulbuch Lambacher schweizer 5 [cite] werden bevor man die Addition und die Subtraktion von natürlichen Zahlen wiederholt werden, die ganzen Zahlen eingeführt. Dies mag auf den ersten Blick etwas verwirrend wirken, wenn zuerst die \glqq neuen \grqq Zahlen eingeführt werden und sich anschließend wieder auf die natürlichen Zahlen zurückgezogen wird. Frau Polz geht in diesem Fall den weg des Schulbuches, um die SuS der 5. Klasse nicht noch weiter zu verwirren und in dem Schulbuch vor und zurück springen zu müssen. Betrachtet man aber hierbei den Lehrplan so wird, ebenso wie im Schulbuch in \textbf{M5 1.1 Natürliche Zahlen und ihre Erweiterung zu den ganzen Zahlen}, die Zahlenbereichserweiterung vor der Wiederholung der Rechenoperationen durchgeführt. (Wieso?)

Die Unterrichtsstunde, welche in dieser Arbeit betrachtet wird, ist dann im nächsten Kapitel \textbf{M5 1.2 Addition und Subtraktion ganzer Zahlen} zu finden. Hierbei sollen 18 Unterrichtsstunden verwendet werden, um die SuS eine Reihe von Kompetenzen und Inhalten zu vermitteln: 

\begin{itemize}
\item wenden die bereits in der Grundschule erlernten schriftlichen Rechenverfahren der Addition und der Subtraktion natürlicher Zahlen auch auf natürliche Zahlen größer als eine Million automatisiert an. Ihre Ergebnisse überprüfen sie durch Abschätzen der Größenordnung kritisch. 
\item bestimmen die Werte von Summen und Differenzen ganzer Zahlen, veranschaulichen ihre Strategien (z. B. mithilfe von Guthaben und Schulden) und erläutern diese; bei angemessen gewählten Zahlen berechnen sie die Werte von Summen und Differenzen auch im Kopf. Sie unterscheiden dabei klar zwischen Vor- und Rechenzeichen. 
\item lösen Gleichungen der Form a + x = b, x - a = b und a - x = b wie in der Grundschule angebahnt, durch systematisches Probieren oder durch Bildung der jeweiligen Umkehraufgabe. 
\item erkennen und nutzen Rechenvorteile, die sich durch Anwenden von Kommutativ- und Assoziativgesetz ergeben; sie verwenden dabei auch, dass jede Differenz als Summe aufgefasst werden kann. 
\item erkennen die Struktur von Termen, die durch Addition und Subtraktion ganzer Zahlen sowie durch Klammersetzung entstehen, gliedern solche Terme unter Verwendung der entsprechenden Fachbegriffe und ermitteln deren Wert in fortlaufender, klar strukturierter Rechnung. 
\end{itemize}\cite{LehrplanGymnasium}

In dem ersten Punkt finden wir unsere Unterrichtsstunde wieder. Diese erlernten Kentnisse, der Rechneregeln auf natürlichen Zahlen sollen im Laufe der Unterrichtssequenz M5 1.2 auf die ganzen erweitert werden. Ebenso wird in der vorliegenden Unterrichtsstunde auf den letzten Punkt eingeganen, da im Häfteintrag die Fachbegriffe der Addition und der Subtraktion festgehalten werden. 