\chapter{Analyse der individuellen und soziokulturellen Voraussetzungen}

\section{Zusammensetzung der Klasse}
Die Klasse 5a besteht aus 16 Schülern und und 18 Schülerinnen, also insgesamt aus 32 Kindern. Alle Kinder der Klasse beherschen gutes Deutsch, somit gibt es kaum sprachliche Barieren. Die noch sehr jungen SuS benötigen eine regelrechte Einführung in das Schulleben am Gymnasium. Dies bedeutet, dass ein Großteil der Stunde verlohren gehen kann, weil man sich mit Fragen wie: \glqq Ich habe nicht die gleiche Farbe wie Sie, was soll ich jetzt tuen? \grqq. Hierbei wurde zum einen eine gesamte Schulstunde darauf verwendet, wie mit Formatierungen an der Tafel (Farben, Unterstreichen, Häftorientierung) umgegangen werden soll, als auch einzelne SuS persönlich gebeten nur noch relevante Unterrichtsbeiträge in das Unterrichtsgespräch einzubringen.\\
\section{Lernklima und Interaktionsverhalten}
Die SuS sind sehr motiviert am Unterrichtsgeschehen teilhaben zu können. Die Lehrerin Frau Polz spielt mit ihnen jede Stunde ein kleines Tafelspiel, wobei zwei Striche und ein Kreis an die Tafel gemalt werden. Wenn die Klasse unruhiger wird kommt erst der eine Strich, dann der andere Strich weg. Sollte es danach nochmals unruhig werden kommt der Kreis auch weg. Sollte es die Klasse schaffen den Kreis bis zum Ende der Stunde zu behalten, kriegen sie weniger oder keine Hausaufgaben. Dieses Spiel wird von der Klasse sehr ernst genommen und die SuS ermahnen sich regelmäßig untereinander ruhig zu sein. Dieses Spiel garantiert in dieser Klasse, dass das Lernklima meistens vorbildlich ist. \\
In den Arbeitsphasen ist des den SuS nicht gestattet sich mit ihrem Nachbarn zu unterhalten, um außerschulischen Gesprächen vorzubeugen. Es ist den SuS aber gestattet aufzustehen und jemand anderen aus der Klasse um Hilfe zu bitten, wenn man bei einer Aufgabe nicht weiter kommt. Dieses Angebot wird von wenigen Lernenden genutzt und kaum missbraucht, meist gehen die Fragen an die Lehrkraft. 
\section {Lernumgebung}
Im Klassenzimmer befindent zusätzlich zur normalen Tafel eine Dokumentenkammera mit Beamer, welche bei den hohen Wänden sehr gut auch über der Tafel genutzt werden kann. Somit können Bilder oder Ähnliches über die Dokumentenkammera gleichzeitig mit dem Tafelbild den SuS vorgeführt werden. Die SuS haben an der hinteren Wand vom Klassenzimmer ein Profil von sich selbst aufgehangen. Diese Profil zeigt was das betreffende Kind gerne in der Freizeig unternimmt, das Alter und die Lieblingsspeise. Dies steigert die Verbundenheit zur Lernumgebung. (Studie?)