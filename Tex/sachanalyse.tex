\chapter{Sachanalyse}
Da wir uns im Zahlenbereich der natürlichen Zahlen befinden spielen in diesme Kontext die Peano-Axiome  sicherlich einen Rolle. Mit diesen Axiomen können die natürlichen Zahlen definiert werden. Diese können mit formalisiert angegeben werden:
\begin{enumerate}
\item $ 0\in \mathbb {N} $ 
\item $\forall n (n\in \mathbb {N} \Rightarrow n^\prime \in \mathbb {N})$ 
\item $\forall n (n\in \mathbb {N} \Rightarrow n^\prime \neq 0)$ 
\item $\forall n,m(m,n\in \mathbb {N}\Rightarrow(m^\prime = n^\prime \Rightarrow m = n))$ 
\item $\forall X (0 \in X \wedge \forall n (n \in \mathbb {N} \Rightarrow (n \in X \Rightarrow n^\prime \in X)) \Rightarrow \mathbb {N} \subseteq X)$
\end{enumerate}(\cite{forster2004analysis})

oder Ausformuliert: 

\begin{enumerate}
\item 0 ist eine natürliche Zahl 
\item jede natürliche Zahl n hat eine natürliche Zahl $n^\prime$ als Nachfolger. 
\item 0 ist kein Nachfolger einer natürlichen Zahl. 
\item Natürliche Zahlen mit gleichem Nachfolger sind gleich. 
\item Enthält X die 0 und mit jeder natürlichen Zahl n auch deren Nachfolger $n^\prime$, so bilden die natürlichen Zahlen e
ine Teilmenge von X
\end{enumerate}(\cite{forster2004analysis})


Bei den natürlichen Zahlen wird leider nicht eindeutig festgelegt, ob die 0 bereits eine natürliche Zahl ist. Dies ist aber ein sehr wichtiger Punkt, welcher nicht vernachtlässigt werden sollte. In dem verwendeten Schulbuch ist die Null eine natürliche Zahl. Dieser Punkt ist für die fachmathematische Seite relevant, da somit die Addition und die Subtranktion ein neutrales Element besitzt. Für die SuS ist es wichtig, das bei Rechenoperationen meist eine Identitätsabbildung möglich ist. Schließt man die Null von den natürlichen Zahlen aus so ist dies nicht mehr möglich. \\
Bei der Betrachtung von neutralen Elementen, drängt es sich von fachmathematischer Seite auf, sich auch Gedanken über inverse Elemente zu machen. Leider ist im Kontext der natürlichen Zahlen dies meist nicht möglich. Außer der Null, welche zu sich selbst invers ist gibt es keine Zahl welche $a + \bar
{a} = e$ oder $a - \bar
{a} = e$ erfüllt. Hierbei ist mit $ \bar
{a}$ das inverse Element gemeint und mit e das neutrale Element. 