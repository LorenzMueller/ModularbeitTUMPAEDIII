\chapter{Artikulationsschema}


\begin{tabularx}{\textwidth}{|X|X|X|X|X|}
\hline
\textbf{Zeit/Sozialform} & \textbf{Geplantes Lehrerverhalten}& \textbf{Geplantes Schülerverhalten} & \textbf{Medien, Materialien} & \textbf{Anmerkungen}\\
\hline
\hline
\multicolumn{5}{|l|}{Wiederholung der letzten Stunde}\\
\hline
5 Minuten, Plenum        &
Der Lehrer kontroliert die Hausaufabe, welche von einem SoS vorgestellt wird. &
Die SuS kontrollieren, ob sie ihre Hausaufgabe richtig bearbeitet haben. Ebenso können bei unterschiedlichen Antworten die Lehrkraft zu Rate gezogen werden, wieso hieber Unterschiede vorhanden sind &
Dokumenten- kammera, Hausaufgabe der letzten Stunde & 7\\
\hline
\multicolumn{5}{|l|}{Motivationsphase (Einstieg und Hinführung)}\\
\hline
7 Minuten, Plenum &
Der Lehrer legt die Abbildung des Baumes und der Schneke unter die Dokumenten- kammera. Hierbei wird den SuS der Sachzusammenhang erklärt und ein SoS gebeten die Bewegung zu veranschaulichen. &
Ein SoS präsentiert ihre Lösung den anderen SuS, die restlichen SuS verfolgen den Verlauf der Schneke und helfen bei der Lösung mit. &
Dokumenten- kammera, Baum- und Schnekenabbilung  & 7\\
\hline
\multicolumn{5}{|l|}{Wiedeholungsphase (durch einfache Beispiele)}\\
\hline
5 Minuten, Plenum &
Die Lehrkraft berechnent mit den SuS eine Addition und eine Subtraktion einer dreistelligen Zahl. &
Die SuS erinnern sich an die oder das gelernte Verfahren, aus der Grundschule, welches sie anwenden können um eine Additon oder eine Subtraktion schriftlich untereinander durchzuführen &
Tafel & 7\\
\multicolumn{5}{|l|}{Verallgemeinerungs- und Ergebnissicherungsphase }\\
\hline
10 Minuten, Plenum &
Der Lehrer schreibt den Häfteintrag an die Tafel und veranschaulicht den Zusammenhang zwischen dem Einführungsbeispiel der Schneke mit dem Häfteintrag&
Die SuS schreiben den Häfteintrag ab. Ebenso sollte durch den Häfteintrag mit der Verbindung der Wiederholungs und der Motivationsphase ein besseres Verständniss für den Zusammenhang zwischen Zahlenstrahl und einfacher Rechung entstanden sein. Zusätzlich werden Fachbegriffe der Addition und der Subtraktion gelernt. &
Tafel, Häfteintrag & 7\\
\multicolumn{5}{|l|}{Anwendungsphase}\\
\hline
13 Minuten, Partnerarbeit &
Der Lehrer erklärt die Aufgabenstellung und unterstützt die SuS bei Fragen.&
Die SuS bearbeiten mit ihrem Nachbarn zusammen die gegebenen Aufgaben. Hierbei werden sowohl die Fachbegriffe  als auch die Rechentechniken vertieft.&
Tafel, Buch & 7\\
\multicolumn{5}{|l|}{Sicherungsphase}\\
\hline
5 Minuten, Plenum &
Der Lehrer lässt einzelne SuS zusammenfassen, was Thema der Stunde war und assistiert hierbei.&
Die SuS festigen ihr, in dieser Stunde erworbenens Wissen, durch die Wiederholung im Plenum &
 & 7\\
\end{tabularx}

