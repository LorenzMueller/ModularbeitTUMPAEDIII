\chapter{Lernziele und Kompetenzen}

Die in der Sekundarstufe I erworbenen Kompetenzen sind unverzichtbare Grundlage für die Arbeit in der Sekundarstufe II. Sie werden dort beständig vertieft und erweitert und können damit auch Gegenstand  der Abiturprüfung sein.(\cite{kmk12})\\
Somit müssen diese Kompetenzen am Anfang der Sekundarstufe II zusammengefasst und gefestigt werden. Dieser Vorgang ist in der forliegenden Stunde der Kerngedanke, da die Inhalte, welche in der Stunde behandelt werden aus der Grundschule (Sekundarstufe I) bekannt sein sollten. Dennoch ist es wichtig diese Kompetenzen zu wiederholen und zusammenzufassen, um möglichst viele Fehlvorstellungen der SuS zu eleminieren. Dies ist in besonderem Maße wichtig, da diese Kompetenzen in jeder Jahrgangstufe benötigt werden und somit als absolute Grundbausteine der mathematischen Kompetenzen gelten.
\\
\\
Die in der Kultusministerkonferenz beschossenen Leitideen ziehen sich Spiralförmig durch das gesamte mathematische Curriculum der Sekundarstufe II. Im vordergrund der beobachteten Unterrichtseinheit steht in diesem Fall die Leitidee \textbf{Algorithmus und Zahl}.
\\
\\
Im Folgenden wird sich einer Betrachtung der geförderten Kompetenzen dieser Unterrichtstunde zugewant. Startet man bei \textbf{(K1) mathematisch argumentieren}, so wird diese Kompetenz in der forliegenden Unterrichtstunde kaum vertieft, dies ist aber den zusammenfassenden Charakter der Stunde geschuldet.\\
Im Kompetenzbereich \textbf{(K2) Probleme mathematisch lösen} bewegt sich der Einführungsteil der Stunde. In diesem Teil wird das Problem \glqq Wann erreicht die Schneke den Ast? \grqq mathematisch iterativ gelöst.\\
In soeben beleuchteten Unterrichtsphase wird ebenso die Kompetenz \textbf{(K3) mathematisch modellieren} angesprochen. Hierbei soll eine wird eine bearbeitete Realsituation von den SuS strukturiert und vereinfacht werden.\\
Die Kompetenz \textbf{(K4) mathematische Darstellungen verwenden} wird in dem Häfteintrag vertieft. Dabei wird eine einfache Addition und Subraktion festgehalten, bei der zusätzlich zur rechnerischen Lösung noch eine graphische Lösung (Zahlenstrahl) gegeben wird. Die graphische Darstellung wird zwar nicht von den SuS selbst entwickelt, soll ihnen aber dennoch die Paralelen aufzeigen.\\
In den Unterrichtsphasen nach dem Häfteintrag und bei der Partnerarbeit werden Additionen und Subtraktionen von den SuS druchgeführt und somit die Kompetenz \textbf{(K5) mit symbolischen, formalen und technischen Elementen der Mathematik umgehen} gefördert.\\
Die SuS sollen im späteren verlauf der Stunde ihre Methode dem Nachbarn erklären und sind somit im Kompetenzbereich \textbf{(K6) mathematisch Kommunizieren}.

\section*{Lernziehle:}

\section*{Die SuS können bis zu vierstellige natürliche Zahlen untereinander addieren und subtrahieren.(Anforderungsbereich I)}


\section*{Die SuS können die Fachbegriffe der Addition und der Subtraktion aufzählen und richtig den entsprechenden Teilen des Terms zuordnen (erster und zweiter Summand, addieren, Addition, Minuend, Subtrahend, subtrahieren und Subtraktion).(Anforderungsbereich I)}